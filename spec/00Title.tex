\thispagestyle{empty}

\noindent
\begin{tabular}{lcr}
{\bf ГОСУДАРСТВЕННЫЙ СТАНДАРТ} & \hspace{3.4cm} & {\bf \draftlogo}\\
{\bf РЕСПУБЛИКИ~БЕЛАРУСЬ} & \\
\end{tabular}

\hrule height 1pt
\vskip0.4mm
\hrule height 2pt

\vskip2cm
\noindent
{\bf\Large Информационные технологии и безопасность}\\[10pt]
{\bf\large КРИПТОГРАФИЧЕСКИЕ АЛГОРИТМЫ НА ОСНОВЕ}\\
{\bf\large SPONGE-ФУНКЦИИ}

\vskip2cm
\noindent
{\bf\Large Iнфармацыйныя тэхналогii i бяспека}\\[10pt]
{\bf\large КРЫПТАГРАФIЧНЫЯ АЛГАРЫТМЫ НА АСНОВЕ}\\
{\bf\large SPONGE-ФУНКЦЫІ}

\vskip9cm
\hrule height 1pt
\vskip0.4mm
\hrule height 2pt
\noindent
\begin{tabular}{p{5cm}cp{4cm}}
\vtop{\null\hbox{{\includegraphics[width=2.6cm]{../figs/stb}}}} & \hspace{6cm} & 
\mbox{}\newline\mbox{}\newline\newline Госстандарт\newline Минск\\
\end{tabular}

\pagebreak

\hrule
\vskip2mm

УДК~004.056.55(083.74)(476)\hfill
МКС~35.240.40\hfill
\mbox{}

\vskip0.5mm
 
{\bf Ключевые слова}: криптографический алгоритм, sponge-функция,
хэширование, шифрование, имитозащита, аутентифицированное шифрование

\vskip0.5mm

\hrule

\rule{0pt}{5mm}

\centerline{\bf Предисловие} 

Цели, основные принципы, положения по государственному регулированию и 
управлению в области технического нормирования и стандартизации 
установлены Законом Республики Беларусь <<О техническом нормировании и 
стандартизации>>.  

\vskip0.2cm

1~РАЗРАБОТАН учреждением Белорусского государственного университета 
<<Науч\-но-исследовательский институт прикладных проблем математики и 
информатики>> 

ВНЕСЕН Комитетом государственной безопасности Республики Беларусь

2~УТВЕРЖДЕН И ВВЕДЕН В ДЕЙСТВИЕ постановлением Государственного комитета 
по стандартизации Республики Беларусь от 05.11.2020~г. \No~64

3~ВЗАМЕН СТБ 34.101.77-2016

\vfill
\hrule
\vskip1mm
Издан на русском языке

\pagebreak
