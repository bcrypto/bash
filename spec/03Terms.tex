\chapter{Термины и определения}

В настоящем стандарте применяют
следующие термины с соответствующими определениями:

{\bf \thedefctr~анонс}:
Входные данные криптографического алгоритма,
которые задают определенный контекст его использования
и могут включать синхропосылку.

{\bf \thedefctr~аутентифицированное шифрование}:
Одновременные шифрование и имитозащита.

{\bf \thedefctr~зашифрование}:
Преобразование сообщения,
направленное на обеспечение его конфиденциальности,
которое выполняется с использованием ключа.

{\bf \thedefctr~имитовставка}:
Двоичное слово, 
которое определяется по сообщению с использованием ключа 
и служит для контроля целостности и подлинности сообщения.

{\bf \thedefctr~имитозащита}:
Контроль целостности и подлинности сообщений, 
который реализуется путем выработки и проверки имитовставок.

{\bf \thedefctr~команда}:
Алгоритм, который связан с автоматом, принимая 
в качестве дополнительного входа его текущее состояние
и возвращая дополнительно обновленное состояние.

{\bf \thedefctr~конфиденциальность}:
Гарантия того, что сообщения доступны для понимания или использования
только тем сторонам, которым они предназначены.

{\bf \thedefctr~(криптографический) автомат}:
Логическое устройство, способное находиться в одном из нескольких 
состояний, выполнять хэширование, шифрование, имитозащиту и другие 
криптографические операции с помощью команд, изменяющих состояние.

{\bf \thedefctr~октет}:
Двоичное слово длины~$8$.

{\bf \thedefctr~подлинность}:
Гарантия того, что сторона действительно является владельцем, 
создателем или отправителем определенного сообщения.

{\bf \thedefctr~псевдослучайные числа}: 
Последовательность элементов, полученная в результате выполнения некоторого
алгоритма и используемая в конкретном случае вместо последовательности случайных
чисел.

{\bf \thedefctr~расшифрование}:
Преобразование, обратное зашифрованию.

{\bf \thedefctr~(секретный) ключ}:
Параметр, который управляет операциями шифрования 
и имитозащиты и который известен только определенным сторонам.

{\bf \thedefctr~синхропосылка}:
Открытые входные данные криптографического алгоритма,
которые обеспечивают уникальность результатов 
криптографического преобразования на фиксированном ключе.

{\bf \thedefctr~случайные числа}: 
Последовательность элементов, каждый из которых не может быть предсказан
(вычислен) только на основе знания предшествующих ему элементов данной
последовательности.

{\bf \thedefctr~снятие защиты}:
Расшифрование и проверка имитовставок.

{\bf \thedefctr~сообщение}:
Двоичное слово конечной длины.

{\bf \thedefctr~установка защиты}:
Зашифрование и вычисление имитовставок.

{\bf \thedefctr~хэш-значение}:
Двоичное слово фиксированной длины, 
которое определяется по сообщению без использования ключа и 
служит для контроля целостности сообщения и для представления 
сообщения в (необратимо) сжатой форме.

{\bf \thedefctr~хэширование}:
Выработка хэш-значений.

{\bf \thedefctr~целостность}:
Гарантия того, что сообщение не изменено 
при хранении или передаче.

{\bf \thedefctr~шифрование}:
Зашифрование или расшифрование.


