\section{Алгоритм \algname{bash-s}}\label{S}

\subsection{Интерфейс}\label{S.IFace}

Значения sponge-функции вычисляются с помощью алгоритма \algname{bash-s}.

Входными данными~\algname{bash-s}
являются слова $W_0,W_1,W_2\in\{0,1\}^{64}$
и числа~$m_1,n_1,m_2,n_2\in\{1,2,\ldots,63\}$.

Выходными данными являются преобразованные слова~$W_0,W_1,W_2$.

\subsection{Переменные}\label{S.Vars}

Используются переменные $T_0,T_1,T_2\in\{0,1\}^{64}$.

\subsection{Алгоритм}\label{S.Alg}

Вычисление~$\algname{bash-s}(W_0,W_1,W_2,m_1,n_1,m_2,n_2)$ выполняется 
следующим образом:
\begin{enumerate}
\item
Установить
$T_0\leftarrow \RotHi^{m_1}(W_0)$.

\item
Установить
$W_0\leftarrow W_0\oplus W_1\oplus W_2$.

\item
Установить
$T_1\leftarrow W_1\oplus \RotHi^{n_1}(W_0)$.

\item
Установить
$W_1\leftarrow T_0\oplus T_1$.

\item
Установить
$W_2\leftarrow W_2\oplus\RotHi^{m_2}(W_2)\oplus\RotHi^{n_2}(T_1)$.

\item
Установить
$T_0\leftarrow \neg W_2$.

\item
Установить
$T_1\leftarrow W_0\vee W_2$.

\item
Установить
$T_2\leftarrow W_0\wedge W_1$.

\item
Установить
$T_0\leftarrow T_0\vee W_1$.

\item
Установить
$W_1\leftarrow W_1\oplus T_1$.

\item
Установить
$W_2\leftarrow W_2\oplus T_2$.

\item
Установить
$W_0\leftarrow W_0\oplus T_0$.

\item
Возвратить~$(W_0,W_1,W_2)$.
\end{enumerate}
