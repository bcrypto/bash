\chapter{Алгоритмы хэширования}\label{HASH}

\section{Интерфейс}\label{HASH.IFace}

Хэширование задается алгоритмом $\algname{bash-hash}$.
Параметром алгоритма является уровень стойкости~$\ell\in\{16,32,\ldots,256\}$ 
(см.~\ref{COMMON.Strength}).

Входными данными $\algname{bash-hash}[\ell]$ является хэшируемое 
сообщение~$X\in\{0,1\}^*$.

Выходными данными является хэш-значение~$Y\in\{0,1\}^{2\ell}$.

Используется алгоритм~$\algname{bash-f}$, определенный в~\ref{F}.

\section{Вспомогательные переменные}\label{HASH.Vars}

Используется переменная~$S\in\{0,1\}^{1536}$.

\section{Алгоритм}\label{HASH.Alg}

Хэширование~$\algname{bash-hash}[\ell](X)$ выполняется следующим образом:
\begin{enumerate}
\item
Определить
$(X_1,X_2,\ldots,X_n)=\Split(X\parallel 01, 1536-4\ell)$.

\item
Установить 
$X_n\leftarrow X_n \parallel 0^{1536-4\ell-|X_n|}$.

\item
Установить
$S\leftarrow 0^{1472}\parallel\itob{\ell/4}_{64}$.

\item
Для $i=1,2,\ldots,n$ выполнить:
\begin{enumerate}
\item
$S[\dots 1536-4\ell)\leftarrow X_i$;
\item
$S\leftarrow\algname{bash-f}(S)$.
\end{enumerate}

\item
Установить
$Y\leftarrow S[\dots 2\ell)$.

\item
Возвратить $Y$.
\end{enumerate}
