\section{Команда \algname{squeeze}}\label{PRG.Squeeze}

\subsection{Интерфейс}\label{PRG.Squeeze.IFace}

Входными данными команды~$\algname{squeeze}$ является длина выхода~$n$.
Число~$n$ должно быть неотрицательным целым, кратным~$8$.

Выходными данными является слово~$Y\in\{0,1\}^n$.

Используется алгоритм~$\algname{bash-f}$, определенный в~\ref{F}.

\subsection{Алгоритм}\label{PRG.Squeeze.Alg}

Команда $\algname{squeeze}(n)$ выполняется следующим образом:
\begin{enumerate}
\item
Выполнить
$\algname{commit}(\texttt{OUT})$.
\item
Установить
$Y\leftarrow\perp$.
\item
Пока $|Y|+r\leq n$:
\begin{enumerate}
\item
$Y\leftarrow Y\parallel S[\dots r)$;
\item
$S\leftarrow\algname{bash-f}(S)$.
\end{enumerate}
\item
Установить
$pos\leftarrow n-|Y|$.
\item
Установить
$Y\leftarrow Y\parallel S[\dots pos)$.
\item
Возвратить~$Y$.
\end{enumerate}

