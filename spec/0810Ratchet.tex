\section{Команда \algname{ratchet}}\label{PRG.Ratchet}

\subsection{Интерфейс}\label{PRG.Ratchet.IFace}

Команда~$\algname{ratchet}$ изменяет состояние автомата,
не принимая и не возвращая данные.

\subsection{Переменные}\label{PRG.Ratchet.Vars}

Используется переменная $T\in\{0,1\}^{1536}$.

\subsection{Алгоритм}\label{PRG.Ratchet.Alg}

Команда $\algname{ratchet}(\perp)$ выполняется следующим образом:
\begin{enumerate}
\item
Установить
$T\leftarrow S$.
\item
Выполнить
$\algname{commit}(\texttt{NULL})$.
\item
Установить
$S\leftarrow S\oplus T$.
%
% Другие варианты: 
% - $S[\dots k)\leftarrow 0^k$, $k\in\{\ell,2\ell,r\}$;
% - $S[\dots r)\leftarrow T[\dots r)$.
%
\end{enumerate}
