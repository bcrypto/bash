\section{Программирование}\label{PRG.Rules}

Работа автомата начинается с вызова команды~$\algname{start}$,
через которую загружаются анонс~$A$ и ключ~$K$. 
%
Позже с помощью команды $\algname{restart}$ можно загрузить новые пары~$(A,K)$.

\if0
Например, загружаются суточные ключи. Или состояние автомата
разветвляется для обслуживания различных направлений связи, 
и по каждой ветке загружается соответствующий анонс. 
\fi

Если в~\algname{start} и во всех вызовах~\algname{restart} ключ был пустым словом, 
то есть не загружался, то автомат находится в бесключевом режиме.
%
В этом режиме:
\begin{itemize}
\item
запрещено вызывать команду \algname{encrypt} (и соответственно \algname{decrypt});
\item
данные, возвращаемые \algname{squeeze}, могут использоваться только 
в качестве хэш-значений.
\end{itemize}

В ключевом режиме синхропосылка загружается явно, как часть анонса, с 
помощью~\algname{start}~/ \algname{restart} или неявно, как волатильные 
открытые данные, с помощью~\algname{absorb}.

Команда \algname{ratchet} вызывается для того, чтобы затруднить возврат к
предыдущим состояниям автомата. Даже если состояние автомата после вызова
\algname{ratchet} раскрыто, криптографические операции до вызова сохраняют
надежность. В частности, затруднено расшифрование ранее зашифрованных данных,
то есть обеспечивается защита от <<чтения назад>>.

В записи программируемого алгоритма фигурирует один или несколько автоматов.
Автоматам назначаются произвольные имена. Команда автомата указывается после 
имени автомата и точки. 

Пример программы c автоматами $\alpha$, $\beta$ и~$\gamma$:
\begin{enumerate}
\item
Выполнить $\alpha.\algname{start}[256, 2](\perp,K)$.
\item
Выполнить $\alpha.\algname{absorb}(I)$.
\item
Выполнить~$\alpha.\algname{ratchet}(\perp)$.
\item
Вычислить~$K_1\leftarrow\alpha.\algname{squeeze}(128)$.
\item
Выполнить $\beta.\algname{start}[128, 1](A_1, K_1)$.
\item
Установить $\gamma\leftarrow\beta$.
\item
Выполнить $\gamma.\algname{restart}(A_2,\perp)$.
\item
Вычислить~$Y_1\leftarrow\beta.\algname{encrypt}(X)$.
\item
Вычислить~$Y_2\leftarrow\gamma.\algname{encrypt}(X)$.
\end{enumerate}

Присваивание $\gamma\leftarrow\beta$ означает, что в $\gamma$ переписываются 
состояние и сопровождающие параметры~$\beta$. 
%
После присваивания автоматы становятся эквивалентными. Эквивалентные автоматы 
функционируют одинаково, в частности, одинаково зашифровывают данные. 
Это недопустимо по соображениям безопасности.
%
В примере эквивалентность разрушается вызовом команды \algname{restart}.

