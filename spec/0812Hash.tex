\section{Хэширование}\label{PRG.Hash}

\subsection{Интерфейс}\label{PRG.Hash.IFace}

Хэширование задается алгоритмом~$\algname{bash-prg-hash}$.
Параметрами алгоритма являются уровень стойкости $\ell\in\{128,192,256\}$
и емкость $d\in\{1,2\}$. 

Входными данными~$\algname{bash-prg-hash}[\ell, d]$ являются 
анонс~$A\in\{0,1\}^{32*}$, хэшируемое сообщение~$X\in\{0,1\}^{8*}$ 
и длина хэш-значения~$n$. Длина~$A$ не должна превосходить~$480$.

Выходными данными является хэш-значение $Y\in\{0,1\}^n$.

\subsection{Переменные}\label{PRG.Hash.Vars}

Используется автомат~$\alpha$.

\subsection{Алгоритм}\label{PRG.Hash.Alg}

Хэширование $\algname{bash-prg-hash}[\ell,d](A,X,n)$ выполняется следующим 
образом:
\begin{enumerate}
\item
Выполнить $\alpha.\algname{start}[\ell,d](A,\perp)$.
\item
Выполнить $\alpha.\algname{absorb}(X)$.
\item
Вычислить $Y\leftarrow \alpha.\algname{squeeze}(n)$.
\item
Возвратить $Y$.
\end{enumerate}
