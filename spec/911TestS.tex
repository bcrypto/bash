\hiddensection{Алгоритм \algname{bash-s}}\label{TEST.S}

В таблице~\ref{Table.TEST.S} представлен пример выполнения
алгоритма \algname{bash-s} с входными параметрами
$(m_1,n_1,m_2,n_2)=(8,53,14,1)$.

\begin{table}[H]
\caption{Алгоритм \algname{bash-s}}\label{Table.TEST.S}
\begin{tabular}{|c|c|l|l|}
\hline
Шаг & Слово & Вычисляется как & Значение\\
\hline
\hline
    & $W_0$ & &
$\hex{B194BAC80A08F53B}$\\
    & $W_1$ & &
$\hex{E12BDC1AE28257EC}$\\
    & $W_2$ & &
$\hex{E9DEE72C8F0C0FA6}$\\
%
1   & $T_0$ & $\RotHi^{m_1}(W_0)$ & 
$\hex{3BB194BAC80A08F5}$\\
2   & $W_0$ & $W_0\oplus W_1\oplus W_2$ & 
$\hex{B96181FE6786AD71}$\\
3   & $T_1$ & $W_1\oplus \RotHi^{n_1}(W_0)$ & 
$\hex{CDFB23D652B779DB}$\\
4   & $W_1$ & $T_0\oplus T_1$ & 
$\hex{F64AB76C9ABD712E}$\\
5   & $W_2$ & $W_2\oplus\RotHi^{m_2}(W_2)\oplus\RotHi^{n_2}(T_1)$ & 
$\hex{F1401A7713A9DFD3}$\\
6   & $T_0$ & $\neg W_2$ & 
$\hex{0EBFE588EC56202C}$\\
7   & $T_1$ & $W_0\vee W_2$ & 
$\hex{F9619BFF77AFFFF3}$\\
8   & $T_2$ & $W_0\wedge W_1$ & 
$\hex{B040816C02842120}$\\
9   & $T_0$ & $T_0\vee W_1$ & 
$\hex{FEFFF7ECFEFF712E}$\\
10  & $W_1$ & $W_1\oplus T_1$ & 
$\hex{0F2B2C93ED128EDD}$\\
11  & $W_2$ & $W_2\oplus T_2$ & 
$\hex{41009B1B112DFEF3}$\\
12  & $W_0$ & $W_0\oplus T_0$ & 
$\hex{479E76129979DC5F}$\\
\hline
\end{tabular}
\end{table}

