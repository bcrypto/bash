\hiddensection{Хэширование (программируемые алгоритмы)}\label{TEST.PrgHash}

В таблице~\ref{Table.TEST.PrgHash} представлены примеры хэширования
с помощью алгоритмов~$\algname{bash-prg-hash}[\ell,d]$.
%
В таблице для различных пар~$(\ell,d)$ приводятся хэш-значения 
сообщения~$S[\dots 8m)$, где~$S$~--- первое слово в таблице~\ref{Table.TEST.F}.
%
При хэшировании анонс $A=\perp$, длина хэш-значения $n=2\ell$.

\begin{table}[H]
\caption{Хэширование (программируемые алгоритмы)}\label{Table.TEST.PrgHash}
\begin{tabular}{|c|l|}
\hline
$m$ & Хэш-значение\\
\hline
%
\hline
\multicolumn{2}{|c|}{$(\ell,d)=(128,2)$}\\
\hline
$0$ &
$\hex{36FA075EC15721F2~50B9A641A8CB99A3~33A9EE7BA8586D06~46CBAC3686C03DF3}$\\
\hline
$127$ &
$\hex{C930FF427307420D~A6E4182969AA1FFC~3310179B8A0EDB3E~20BEC285B568BA17}$\\
\hline                                                        
$128$ &
$\hex{92AD1402C2007191~F2F7CFAD6A2F8807~BB0C50F73DFF95EF~1B8AF08504D54007}$\\
\hline                                                        
$150$ &
$\hex{48DB61832CA10090~03BC0D8BDE67893A~9DC683C48A5BC23A~C884EB4613B480A6}$\\
%
\hline
\hline
\multicolumn{2}{|c|}{$(\ell,d)=(192,1)$}\\
\hline
$143$ &
$\hexz{6166032D6713D401~A6BC687CCFFF2E60~3287143A84C78D2C~62C71551E0E2FB2A}$\\
&
 $\hex{F6B799EE33B5DECD~7F62F190B1FBB052}$\\
\hline
$144$ &
$\hexz{8D84C82ECD0AB646~8CC451CFC5EEB3B2~98DFD381D200DA69~FBED5AE67D26BAD5}$\\
&
 $\hex{C727E2652A225BF4~65993043039E338B}$\\
\hline                                                        
$150$ &
$\hexz{47529F9D499AB6AB~8AD72B1754C90C39~E7DA237BEB16CDFC~00FE87934F5AFC11}$\\
&
 $\hex{01862DFA50560F06~2A4DAC859CC13DBC}$\\
\hline
\end{tabular}
\end{table}

