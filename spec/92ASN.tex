\begin{appendix}{Б}{рекомендуемое}{Модуль АСН.1}
\label{ASN}

\mbox{}

В модуле АСН.1 определяются идентификаторы следующих алгоритмов:
\begin{center}
\begin{tabular}{p{3.6cm}p{12.7cm}}
\texttt{bashNNN} &
\addendum{А}лгоритм хэширования $\algname{bash-hash}[l]$;\\
%
\texttt{bash-prg-hashNNND} &
\addendum{А}лгоритм хэширования $\algname{bash-prg-hash}[l,d]$;\\
%
\texttt{bash-prg-aeLLLD} &
\addendum{А}лгоритм аутентифицированного шифрования $\algname{bash-prg-ae}[l,d]$;\\
%
\texttt{bash-f}  &
\addendum{А}лгоритм вычисления значений sponge-функции (пункт~\ref{F}).\\
\end{tabular}
\end{center}

Здесь \texttt{NNN}~--- десятичный код числа~$2l$, 
\texttt{LLL}~--- десятичный код числа~$l$, 
\texttt{D}~--- десятичный код числа~$d$.

Алгоритм $\algname{bash-hash}[l]$ определяет функцию 
хэширования~$h\colon \{0,1\}^*\to\{0,1\}^{2l}$. 
Эта функция может использоваться в алгоритмах 
ЭЦП СТБ~34.101.45, уточняя их.
%
Правила использования~$h$ определены в СТБ 34.101.45 (пункт 5.5).
%
Уточненным алгоритмам ЭЦП присваиваются следующие идентификаторы:
\begin{center}
\begin{tabular}{p{4.4cm}p{11.6cm}}
\texttt{bign-with-bashNNN} &
\addendum{А}лгоритмы ЭЦП СТБ 34.101.45 (пункт 7.1)
c функцией хэширования, заданной алгоритмом~$\algname{bash-hash}[2l]$;\\
%
\texttt{bign-ibs-with-bashNNN} &
\addendum{А}лгоритмы идентификационной ЭЦП СТБ 34.101.45
c функцией хэширования, заданной алгоритмом~$\algname{bash-hash}[2l]$.\\
\end{tabular}
\end{center}

Здесь, как и прежде, \texttt{NNN}~--- десятичный код числа~$2l$.

Модуль АСН.1 имеет следующий вид:

\verbatiminput{bash-module-v2.asn}

\end{appendix}
